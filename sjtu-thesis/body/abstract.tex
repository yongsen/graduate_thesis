%%==================================================
%% abstract.tex for SJTU Master Thesis
%% based on CASthesis
%% modified by wei.jianwen@gmail.com
%% version: 0.3a
%% Encoding: UTF-8
%% last update: Dec 5th, 2010
%%==================================================

\begin{abstract}

近年来,无线网络尤其是移动网络发展迅速,一方面造成业务流量和用户需求的增加,另一方面带来频谱利用率的下降,因此如何通过有效的资源分配实现系统可靠性与网络传输性能的平衡成为移动网络的关键问题。传统的频谱资源分配一般采用理论上的无线传播或链路质量模型,但是这些模型由于忽略了多径效应和移动性等因素,在实际应用过程中通常难以实现有效的资源调度与分配。实际应用中一般利用实测数据对当前网络状态进行评估,同时实现无线资源的动态分配,如功率控制、速率适配及接入控制等,因此实时准确的通信质量测试对于移动网络的可靠稳定运行至关重要。移动无线网络通信质量最重要的两项指标为信道状态与链路质量,分别表征了其物理层与链路层的通信质量信息,同时也是网络运行与优化的重要参数,因此两者的准确高效的测量能实现移动网络可靠性与传输性能的有效平衡。对于移动无线网络通信质量测试而言,主要问题是如何在不同的网络场景与系统需求下,实现测试精度与开销之间的有效平衡。

首先,由于GSM-R网络对安全性具有严格要求,现有测试方法均采用高频采样,成本较高且仅适用于离线测试,无法应用于在线测试及资源调度。对于高速移动网络GSM-R而言,在满足测量精度的前提先应尽量降低其测试开销,实现对GSM-R网络的在线实时测试,保证整个高铁系统的稳定可靠运行。而GSM-R网络具有移动终端高速移动性和无线传播环境复杂性的特点,从而对其通信质量测试形成巨大挑战。传统的信号强度测试算法无法直接应用于GSM-R网络测试中,这就要求GSM-R网络信号强度测试算法必须具有实时高效的特点,以适应高速铁路的特殊要求,即在高速移动条件下满足信道状态测试精度并降低测试开销。

其次,对于802.11n网络而言,除了移动性及无线传播环境的影响外,MIMO-OFDM技术进一步增加了链路质量测试与建模的复杂度。MIMO-OFDM技术一方面显著地提升了网络性能,另一方面给802.11n网络链路质量测试与建模带来新的问题:获得所有配置下的链路质量模型需要进行更多的探测与采样,同时MIMO-OFDM配置造成链路质量模型的过渡窗口效应,从而严重降低链路质量预测精度。因此如何针对802.11n网络的移动性与多配置性,设计动态的链路质量测试与建模算法,以准确刻画MIMO-OFDM配置的链路质量,从而根据当前网络状态进一步提升其网络性能,即在MIMO-OFDM多配置情况下实现不同配置链路质量模型的在线实时测试与建模。

本文主要针对高速移动网络和无线局域网络中的信道估计与链路测试问题,对移动网络的通信质量测试进行详细分析,针对高速移动特性及MIMO-OFDM多配置特性分别提出信道状态动态测试算法及链路质量在线测试与建模框架,通过当前网络状态实时调整测试参数,在满足测试精度的前提下降低测试开销,最后通过算法设计、系统实现与实验测试对移动网络通信质量测试算法进行性能评估。


  \keywords{移动无线网络,通信质量测试,GSM-R,802.11n,MIMO-OFDM,信道状态估计,链路质量建模}
\end{abstract}

\begin{englishabstract}

The mobile wireless networks have experienced rapid development in recent years, and the growth is expected to continue unabated. For different types of mobile networks, a basic consideration is the accurate channel state estimation and link quality measurement to get efficient trade-off between reliability and data rate. For GSM-R networks deployed for communications between train and railway regulation control centers in high-speed railway, it requires real-time measurement to ensure the reliability of the system \cite{baldini2010early}. At the same time, it is necessary to make dynamic measurement due to the complexity of the radio propagation environments and the varied terrains along the high-speed railway route. On the other hand, the continued success of mobile 802.11n depends on their ability to efficiently configure different PHY/MAC enhancements. This is challenging in that multi-configuration in mobile 802.11n not only requires far more samples to acquire sufficient information for all possible channel settings, but also introduces significant complications in channel modeling. Furthermore, channels are more vulnerable to environmental variability and terminal mobility in mobile 802.11n. Therefore, accurate channel measurement and prediction is becoming increasingly important in mobile wireless networks, and it is crucial to lower the estimation overhead to address the issues of mobility and multi-configuration so that real-time measurement can be implemented to ensure the reliability or data rate.

For the channel state estimation in mobile networks, Lee's method proposed a standard procedure of local average power estimation, which determined the proper length and required sampling numbers for estimating the local average in the case of Rayleigh fading channels \cite{lee1985estimate}. Velocity adaptive handoff algorithms \cite{Austin1994} get the amount of spatial averaging required for local mean estimation of Rician fading according to Lee's standard procedure by approximation, but it has too high overhead to be applied in real-time measurement. The Generalized Lee method \cite{Vega2009} allows estimating the mean values without the requirement of a priori knowing the distribution function, which is based on measured field data samples, but the optimum length of averaging interval is calculated using all the routes of the database with high overhead. This paper combines Lee's method and EM algorithm to estimate the Rician fading channels in GSM-R networks. The basic procedure is same to the Lee's method of local mean power estimation, except that the multi-path fading is Rician distributed. This method takes advantage of the sampling signals and Rician fading parameters of last estimation to improve estimation accuracy and reduce measurement overhead. The determination of proper length of statistical interval and required number of averaging samples are adaptive to different propagation environments.

Furthermore, recent studies show that Received Signal Strength (RSS) is a weak indicator for 802.11n channel quality due to the large transition window with respect to Packet Delivery Ratio (PDR), and there exists a fundamental and inevitable tradeoff between the accuracy and overhead in channel measurement and prediction. This is further complicated by the distinctive features in mobile 802.11n networks, specifically, multiple PHY/MAC settings and spatial-temporal variation channels. In this work, we present an on-line PDR-RSS modeling framework for mobile 802.11n networks. It incorporates a novel design by exploiting both packet-level and physical-level metrics, along with the diversity property of multi-configuration simultaneously to overcome channel capturing problem in the existing PDR-RSS models. This on-line framework also strikes a balance between the measurement overhead and accuracy. We further develop a rate adaption algorithm to advocate the advantage of on-line PDR-RSS modeling framework. It adopts an on-line rate selection process with high precision. Through a real world implementation on our testbed, we evaluate the proposed rate adaption algorithm over different scenarios and routes. The experimental results indicate that it can achieve throughput gains up to 40\% over the Minstrel rate control algorithm under different MIMO configurations.

  \englishkeywords{Mobile wireless network, performance measurement, GSM-R, 802.11n, MIMO-OFDM, local mean power estimation, packet delivery modeling}
\end{englishabstract}
