%%==================================================
%% abstract.tex for SJTU Master Thesis
%% based on CASthesis
%% modified by wei.jianwen@gmail.com
%% version: 0.3a
%% Encoding: UTF-8
%% last update: Dec 5th, 2010
%%==================================================

\begin{abstract}
安全始终是高速铁路发展过程中的首要目标,而GSM-R网络则是保证高铁系统安全性的重要环节,因此有必要对GSM-R网络进行在线实时测试,在确保GSM-R网络正常通信的基础之上,保证整个高铁系统的稳定运行。但是由于高铁系统具有较为特殊的无线传播环境,传统的信号强度测试算法无法直接应用于GSM-R网络测试中,例如高铁线路周围地形复杂多变以及列车处于高速运行状态,这就要求GSM-R网络信号强度测试算法必须具有实时高效的特点,以适应高速铁路的特殊要求。

  \keywords{\large 上海交大 \quad 饮水思源 \quad 爱国荣校}
\end{abstract}

\begin{englishabstract}

Recent studies show that Received Signal Strength (RSS) is a weak indicator for 802.11n channel quality due to the large transition window with respect to Packet Delivery Ratio (PDR), and there exists a fundamental and inevitable tradeoff between the accuracy and overhead in channel measurement and prediction. This is further complicated by the distinctive features in mobile 802.11n networks, specifically, multiple PHY/MAC settings and spatial-temporal variation channels. In this work, we present an online PDR-RSS modeling framework for mobile 802.11n networks. The proposed online PDR-RSS model incorporates a novel design by exploiting both packet-level and physical-level metrics, along with the diversity property of multi-configuration simultaneously to overcome channel capturing problem in the existing PDR-RSS models. This online framwork also strikes a balance between the measurement overhead and accuracy. We further develop a rate adaption algorithm, Graded Rate or GradedR, to advocate the advantage of  online PDR-RSS modeling framework. GradedR adopts an online rate selection process with high precision. Through a real world implementation on our testbed, we evaluate the GradedR over different scenarios and routes. The experimental results indicate that GradedR can achieve throughput gains up to 40\% over the Minstrel rate control algorithm under different MIMO configurations.

The on-line and dynamic estimation algorithm for Rician fading channels in GSM-R networks is proposed, which is an expansion of local mean power estimation of Rayleigh fading channels. The proper length of statistical interval and required number of averaging samples are determined which are adaptive to different propagation environments. It takes advantage of the sampling signals and Rician fading parameters of last estimation to reduce measurement overhead. The performance of this method was evaluated by measurement experiment along the Beijing-Shanghai high-speed railway. When it is NLOS propagation, the required sampling intervals can be increased from $1.1\lambda$ in Lee's method to $3.7\lambda$ of the on-line and dynamic algorithm. The sampling interval can be set up to 12$\lambda$ although the length of statistical interval decreases when there is LOS signal, which can reduce the measurement overhead significantly. The algorithm can be applied in coverage assessment with lower measurement overhead, and in dynamic and adaptive allocation of wireless resource.

  \englishkeywords{\large GSM-R, 802.11n, local mean power estimation, packet delivery modeling}
\end{englishabstract}
