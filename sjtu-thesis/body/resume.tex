%%==================================================
%% resume.tex for SJTU Master Thesis
%% based on CASthesis
%% modified by wei.jianwen@gmail.com
%% version: 0.3a
%% Encoding: UTF-8
%% last update: Dec 5th, 2010
%%==================================================

\begin{resume}

\begin{resumesection}{教育背景}
%\cventry{2010--2012}{工学硕士}{上海交通大学}{上海}{}{硕士课题:移动无线网络测试与建模研究,主要工作包括平台搭建及软件开发}
%\cventry{2006--2010}{工学学士}{山东大学}{济南}{}{毕业设计:ZigBee网络广播通信协议性能评估研究,成绩:96;专业成绩:91/100}
\end{resumesection}

\begin{resumesection}{社会经历}
%\cventry{2010--2012}{电子信息与电气工程学院}{上海交通大学}{上海}{}{}
\begin{itemize}
\item \textbf{科研工作},Smart Wireless System Group
  \begin{itemize}
    \item 开发GSM-R网络空中接口测试系统平台,运行于PC104平台及Windows XP Embedded操作系统,基于C\#与Microsoft .NET Compact Framework,并于京沪高铁进行通信测试与数据采集。
    \item 开发移动802.11n网络通信性能测试软件,运行于高通Atheros WiFi设备及Linux操作系统,基于Linux C与Linux无线驱动\texttt{ath9k},并于实验室与宿舍环境下进行测试。
  \end{itemize}
\item \textbf{助研}
  \begin{itemize}
    \item 参与科研项目的申请、执行与结题:申请报告,硬件选型,软件开发,实验测试,结题报告
    \begin{itemize}
      \item 国家自然科学基金“认知无线网络动态频谱拍卖机制与优化算法研究”
      \item 国家自然科学基金“智能电网中需求响应与能量有效协同优化与控制”
      \item 铁道部重点项目“GSM-R网络通信质量测试技术研究”
      %(NO. 61172064)(NO. 61104091)(2010X020)
    \end{itemize}
    \item 会议及期刊论文评审经历,本科生PRP(Participation in Research Program)项目指导
  \end{itemize}
\item \textbf{助教}
  \begin{itemize}
      \item 本科生课程:数字控制系统PLC
  \end{itemize}
\item \textbf{副部长}, 电院研会科技部
  \begin{itemize}
    \item 负责组织学术与科技活动,包括学科导航,专利讲座,学术沙龙,专家论坛等
  \end{itemize}
\item \textbf{志愿者}
  \begin{itemize}
  \item 上海世博会Expo后勤志愿者,2010.9
  \item CSNC(China Satellite Navigation Conference)会议注册及会场引领志愿者,2011.5
  \end{itemize}
\end{itemize}
%\cventry{2006--2010}{控制科学与工程学院}{山东大学}{济南}{}{}
\begin{itemize}
  \item 可测液体温度的数字温度计,基于PIC单片机
  \item 基于飞思卡尔单片机及激光对管的智能车
  \item 基于NS2的Zigbee网络通信性能测试与分析
\end{itemize}
\end{resumesection}

\begin{resumesection}{获奖情况}
\begin{description}
  \item[2006--2007] 山东大学优秀学生奖学金,10\%;山东大学暑期社会实践优秀奖
  \item[2007--2008] 山东大学暑期社会实践团体二等奖;山东大学科技创新大赛三等奖
  \item[2008--2009] 山东大学优秀学生奖学金,5\%;山东省大学生智能车大赛三等奖
\end{description}
\end{resumesection}

\begin{resumesection}{研究兴趣}
\begin{description}
  \item[理论] GSM/GSM-R,Zigbee,无线局域网络,移动无线网络,无线频谱分配与速率控制;
  \item[应用] 能量有效性算法与跨层协议设计,无线链路质量建模,Linux驱动与移动应用开发。
\end{description}
\end{resumesection}

\begin{resumesection}{掌握技能}
\begin{description}
  \item[硬件] PC104/PC104+平台,Atheros/Broadcom无线设备,GSM/GSM-R模块。
  \item[软件] NS2/NS3,Wireshark,iperf;Mathematic,Matlab,Gnuplot;Visio,\LaTeX。
  \item[语言] C\#,C++,XML,HTML;Linux C,Tcl/Otcl,awk/Gawk,Linux shell。
  \item[开发] Microsoft .NET Compact Framework,Visual Studio;ath9k,Madwifi,mac80211。
\end{description}
\end{resumesection}

\begin{resumesection}{其他}
\begin{description}
  \item[托福] 总分90;阅读25,听力23,口语18,写作24
  \item[证书] 全国计算机等级考试三级,网络技术
  \item[爱好] 足球,乒乓球,羽毛球,单车,摄影
\end{description}
\end{resumesection}

\begin{resumesection}{基本情况}
xxx,男,上海人,1985 年~12 月出生,未婚,
上海交通大学物理系在读博士研究生。
\end{resumesection}

\begin{resumelist}{教育状况}
XXXX 年~9 月至~XXXX 年~7 月,上海交通大学, 本科,专业:XXXX

XXXX 年~9 月至~XXXX 年~7 月,上海交通大学, 硕士研究生,专业:XXXX

XXXX 年~9 月至~XXXX 年~7 月,上海交通大学,
博士研究生(提前攻读博士),专业:XXXX
\end{resumelist}

\begin{resumelist}{工作经历}
无。
\end{resumelist}

\begin{resumelist}{研究兴趣}
XXXXXXX。
\end{resumelist}

\begin{resumelist}{联系方式}
通讯地址:上海市闵行区东川路800号,上海交通大学物理系

邮编:200240

E-mail: abcde@sjtu.edu.cn
\end{resumelist}

\end{resume}
