%%==================================================
%% thanks.tex for SJTU Master Thesis
%% based on CASthesis
%% modified by wei.jianwen@gmail.com
%% version: 0.3a
%% Encoding: UTF-8
%% last update: Dec 5th, 2010
%%==================================================

\begin{thanks}

  逝者如斯!值此毕业论文即将杀青之际,希望能够通过简短的文字,对各位老师、同学、朋友、亲人,以及其他有关的人、事、物,表达最真挚的谢意!
  
  首先衷心感谢我的指导老师龙承念教授,正是在龙老师的悉心指导下,使得我在研究生阶段积累了宝贵的研究经历。龙老师在科学研究上认真严谨,为我们树立了榜样,同时在学术问题的探讨上事无巨细,促使我们塑造良好的科学研究态度。在科研项目的申请、执行与总结的过程中,龙老师积极引导并鼓励我们参与其中,使得我们对问题有更深刻的认识。在学术问题与论文写作方面,龙老师严格要求并竭力指导。所有这些经历都是我今后人生中不可或缺的宝贵财富!
  
  其次我要感谢实验室其他老师的指导与建议,关新平教授、陈彩莲副教授和杨博副教授多次提出宝贵的建议,并提供了良好的学习与实验的环境,从而使我能够更好的投入到学术研究与论文写作之中。同时感谢本文在学院盲审过程中老师提出的修改意见,本文的最终成稿离不开评审老师的批评与指正。
  
  感谢实验室的师兄师姐及其他同学,在与他们的学习与交流过程中,使我学习到许多新的知识与方法,包括通信、控制及编程等方面的内容。同时我要感谢参与PRP项目的各位本科生同学,感谢他们在软件开发、报告总结及论文撰写过程中的辛勤工作!在论文的写作过程中,还得到了ath9k开源社区及其它社区成员的指导与帮助,包括出现在参考文献中各位作者及计算机与通信领域各位前辈的卓越贡献,同时得到了上海交通大学图书馆、ACM、IEEE等机构的资源支持,在此一并致以诚挚的谢意!

  最后感谢我的家人,感谢他们在学业与生活上给予我的支持、理解与关怀,他们是使我不断努力学习的最强大动力!。再次感谢曾经关心与帮助过我的所有人!

\end{thanks}
