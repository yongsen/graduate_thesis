%% start of file `template-zh.tex'.
%% Copyright 2006-2012 Xavier Danaux (xdanaux@gmail.com).
%
% This work may be distributed and/or modified under the
% conditions of the LaTeX Project Public License version 1.3c,
% available at http://www.latex-project.org/lppl/.


%\documentclass[11pt,a4paper,roman]{moderncv}   % possible options include font size ('10pt', '11pt' and '12pt'), paper size ('a4paper', 'letterpaper', 'a5paper', 'legalpaper', 'executivepaper' and 'landscape') and font family ('sans' and 'roman')

%\usepackage[unicode,pdfstartview=FitH]{hyperref}

% moderncv 主题
%\moderncvstyle{banking}                        % 选项参数是 ‘casual’, ‘classic’, ‘oldstyle’ 和 ’banking’
%\moderncvstyle{classic}                        % 选项参数是 ‘casual’, ‘classic’, ‘oldstyle’ 和 ’banking’
%\moderncvstyle{casual}                        % 选项参数是 ‘casual’, ‘classic’, ‘oldstyle’ 和 ’banking’
%\moderncvcolor{blue}                          % 选项参数是 ‘blue’ (默认)、‘orange’、‘green’、‘red’、‘purple’ 和 ‘grey’
%\nopagenumbers{}                             % 消除注释以取消自动页码生成功能

% 字符编码
%\usepackage[utf8]{inputenc}                   % 替换你正在使用的编码
%\usepackage{CJK}
%\usepackage{CJKutf8}                         % if you need to use CJK to typeset your resume in Chinese, Japanese or Korean


% 调整页面出血
%\usepackage[scale=0.8]{geometry}
%\setlength{\hintscolumnwidth}{3cm}           % 如果你希望改变日期栏的宽度

% 个人信息
\firstname{}
\familyname{马 永森}
%\title{简历题目 (可选项)}                       % 可选项、如不需要可删除本行
\address{上海交通大学}{上海市闵行区东川路800号}  % 可选项、如不需要可删除本行
\mobile{(+86) 1880-190-0543}                     % optional, remove the line if not wanted
\phone{(+86) 021-3420-6656}                      % optional, remove the line if not wanted
\email{mayongsen@gmail.com}                          % optional, remove the line if not wanted
\homepage{yongsen.github.com}                  % 可选项、如不需要可删除本行
%\extrainfo{附加信息 (可选项)}                  % 可选项、如不需要可删除本行
\photo[64pt][0.4pt]{profile.pdf}                  % ‘64pt’是图片必须压缩至的高度、‘0.4pt‘是图片边框的宽度 (如不需要可调节至0pt)、’picture‘ 是图片文件的名字;可选项、如不需要可删除本行
%\quote{引言(可选项)}                           % 可选项、如不需要可删除本行

% 显示索引号;仅用于在简历中使用了引言
%\makeatletter
%\renewcommand*{\bibliographyitemlabel}{\@biblabel{\arabic{enumiv}}}
%\makeatother

% 分类索引
%\usepackage{multibib}
%\newcites{book,misc}{{Books},{Patents}}
%----------------------------------------------------------------------------------
%            内容
%----------------------------------------------------------------------------------
\begin{document}
\begin{CJK}{UTF8}{hei}                       % 详情参阅CJK文件包
\maketitle

\section{教育背景}
\cventry{2010--2012}{工学硕士}{上海交通大学}{上海}{}{硕士课题:移动无线网络测试与建模研究,主要工作包括平台搭建及软件开发}
\cventry{2006--2010}{工学学士}{山东大学}{济南}{}{毕业设计:ZigBee网络广播通信协议性能评估研究,成绩:96;专业成绩:91/100}

\section{社会经历}
\cventry{2010--2012}{电子信息与电气工程学院}{上海交通大学}{上海}{}{}
\begin{itemize}
\item \textbf{科研工作},Smart Wireless System Group
  \begin{itemize}
    \item 开发GSM-R网络空中接口测试系统平台,运行于PC104平台及Windows XP Embedded操作系统,基于C\#与Microsoft .NET Compact Framework,并于京沪高铁进行通信测试与数据采集。
    \item 开发移动802.11n网络通信性能测试软件,运行于高通Atheros WiFi设备及Linux操作系统,基于Linux C与Linux无线驱动\texttt{ath9k},并于实验室与宿舍环境下进行测试。
  \end{itemize}
\item \textbf{助研}
  \begin{itemize}
    \item 参与科研项目的申请、执行与结题:申请报告,硬件选型,软件开发,实验测试,结题报告
    \begin{itemize}
      \item 国家自然科学基金“认知无线网络动态频谱拍卖机制与优化算法研究”
      \item 国家自然科学基金“智能电网中需求响应与能量有效协同优化与控制”
      \item 铁道部重点项目“GSM-R网络通信质量测试技术研究”
      %(NO. 61172064)(NO. 61104091)(2010X020)
    \end{itemize}
    \item 会议及期刊论文评审经历,本科生PRP(Participation in Research Program)项目指导
  \end{itemize}
\item \textbf{助教}
  \begin{itemize}
      \item 本科生课程:数字控制系统PLC
  \end{itemize}
\item \textbf{副部长}, 电院研会科技部
  \begin{itemize}
    \item 负责组织学术与科技活动,包括学科导航,专利讲座,学术沙龙,专家论坛等
  \end{itemize}
\item \textbf{志愿者}
  \begin{itemize}
  \item 上海世博会Expo后勤志愿者,2010.9
  \item CSNC(China Satellite Navigation Conference)会议注册及会场引领志愿者,2011.5
  \end{itemize}
\end{itemize}
\cventry{2006--2010}{控制科学与工程学院}{山东大学}{济南}{}{}
\begin{itemize}
  \item 可测液体温度的数字温度计,基于PIC单片机
  \item 基于飞思卡尔单片机及激光对管的智能车
  %\item 基于NS2的Zigbee网络通信性能测试与分析
\end{itemize}
%\cvdoubleitem{高等数学}{96}{大学物理}{98}
%\cvdoubleitem{自动控制原理}{92}{运动控制系统}{93}
%\cvdoubleitem{计算机控制系统}{91}{嵌入式系统设计}{95}
%\cvdoubleitem{信号与系统 (双语)}{89}{毕业设计}{96}
%高等数学:96;大学物理:98;自动控制原理:92;运动控制系统:93;计算机控制系统:91;嵌入式系统设计:95;信号与系统 (双语):89;毕业设计:96
%\section{Honors and Awards}
%\cvdoubleitem{山东大学优秀学生奖学金}{~~~~10\%, \hfil{~~~~2007}}{山东大学暑期社会实践}{~~~~优秀奖, \hfil{2007}}
%\cvdoubleitem{山东大学暑期社会实践团体}{二等奖, \hfil{2008}}{山东大学优秀学生奖学金}{5\%, \hfil{~~~~~~2009}}
%\cvdoubleitem{山东省大学生智能车大赛}{~~~~三等奖, \hfil{2009}}{山东大学科技创新大赛}{~~~~三等奖, \hfil{2009}}
%
%\section{Selected Courses}
%\cvdoubleitem{高等数学}{96}{大学物理}{98}
%\cvdoubleitem{自动控制原理}{92}{运动控制系统}{93}
%\cvdoubleitem{计算机控制系统}{91}{嵌入式系统设计}{95}
%\cvdoubleitem{信号与系统 (双语)}{89}{毕业设计}{96}
%
%\section{获奖情况}
%\cvitem{2006--2007}{山东大学优秀学生奖学金,10\%;山东大学暑期社会实践优秀奖}
%\cvitem{2007--2008}{山东大学暑期社会实践团体二等奖;山东大学科技创新大赛三等奖}
%\cvitem{2008--2009}{山东大学优秀学生奖学金,5\%;山东省大学生智能车大赛三等奖}

\section{获奖情况}
\cvitem{2006--2007}{山东大学优秀学生奖学金,10\%;山东大学暑期社会实践优秀奖}
\cvitem{2007--2008}{山东大学暑期社会实践团体二等奖;山东大学科技创新大赛三等奖}
\cvitem{2008--2009}{山东大学优秀学生奖学金,5\%;山东省大学生智能车大赛三等奖}

\section{课程成绩}
\cvlistdoubleitem{高等数学:96}{大学物理:98}
\cvlistdoubleitem{自动控制原理:92}{运动控制系统:93}
\cvlistdoubleitem{计算机控制系统:91}{嵌入式系统设计:95}
\cvlistdoubleitem{信号与系统 (双语):89}{毕业设计:96}

\section{研究兴趣}
\cvitem{理论}{GSM/GSM-R,Zigbee,无线局域网络,移动无线网络,无线频谱分配与速率控制;}
\cvitem{应用}{能量有效性算法与跨层协议设计,无线链路质量建模,Linux驱动与移动应用开发。}

\section{掌握技能}
\cvitem{硬件}{PC104/PC104+平台,Atheros/Broadcom无线设备,GSM/GSM-R模块。}
\cvitem{软件}{NS2/NS3,Wireshark,iperf;Mathematic,Matlab,Gnuplot;Visio,\LaTeX。}
\cvitem{语言}{C\#,C++,XML,HTML;Linux C,Tcl/Otcl,awk/Gawk,Linux shell。}
\cvitem{开发}{Microsoft .NET Compact Framework,Visual Studio;ath9k,Madwifi,mac80211。}

\section{其他}
\cvitem{托福}{总分90;阅读25,听力23,口语18,写作24}
\cvitem{证书}{全国计算机等级考试三级,网络技术}
\cvitem{爱好}{足球,乒乓球,羽毛球,单车,摄影}

%\section{论文与专利}
%\nocite{Long1304:Online}
%\nocite{Yong1210:Line}
%\nocite{Long1210:Dual}
%\bibliographystyle{unsrt}
%\bibliography{cv}
%
%\nocitemisc{Ma2012gis}
%\nocitemisc{Ma2012dynamic}
%\nocitemisc{Ma2012online}
%\nocitemisc{Ma2011software}
%\bibliographystylemisc{unsrt}
%\bibliographymisc{patent-zh}

%\clearpage
\end{CJK}
\end{document}

%% 文件结尾 `template-zh.tex'.
