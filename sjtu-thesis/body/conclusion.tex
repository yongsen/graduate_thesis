%%==================================================
%% conclusion.tex for SJTU Master Thesis
%% based on CASthesis
%% modified by wei.jianwen@gmail.com
%% version: 0.3a
%% Encoding: UTF-8
%% last update: Dec 5th, 2010
%%==================================================

\chapter*{全文总结\markboth{全文总结}{}}
\addcontentsline{toc}{chapter}{全文总结}

本文针对移动网络中信道状态及链路质量测试问题,提出动态测试算法以根据当前网络状态调整测试参数,在保证测试精度的前提下尽量降低测试开销,从而保证网络的可靠运行及传输性能。

针对GSM-R网络信道状态采样与估计,本文提出接收信号强度动态测试算法,解决高速移动性及传播环境复杂性对信道采样的不利影响,在保证一定测试精度的条件下降低测试开销,并通过系统设计与实现对该算法进行评估。该算法通过采样数据结合衰落参数历史值,对当前衰落参数进行估计,确定不同衰落参数条件下的统计区间与采样点数。在城区、山地、丘陵等密集区域,由于多径衰落现象加重,且直射路径功率所占比例较低,需要进行较为频繁的采样与统计,确保统计区间$2L \leq 20m$,采样间隔$\Delta d \leq 0.3m$;在平原、高架桥等开阔区域,移动台接收功率较大,且一般存在较大比例的直射路径功率,在同样的统计区间内只需做较少的采样,保证统计区间$2L \leq 50m$,采样间隔$\Delta d \leq 1.5m$,便可以满足本地均值的准确性要求。对应列车运行速度在$300km/h$ 时,采样时间间隔为$2.0ms$到$18.0ms$ 时,才能够保证测量数据的可靠性。在实际工程应用中的GSM-R 网络无线覆盖测量,一般采用采样间隔$\Delta d = 4cm$、统计区间$10m \leq 2L \leq 100m$ 的方法,参照本章关于莱斯衰落信道下采样算法的推导,可以在高铁线路中的开阔区域适当提高采样间隔,从而在确保数据可靠性的同时降低测量开销;另一方面针对GPS测距触发方式的测量方法,利用高速铁路列车运行速度相对固定的特点,结合列车运行速度、当前采样数据及衰落参数历史数据,采用时间触发的方式进行采样间隔与统计区间的确定。

对于链路质量测试与建模,本文主要解决移动802.11n网络中MIMO-OFDM配置及移动性对链路质量测试与建模带来的问题,提出在线测试与建模框架,在保证系统可靠性的前提下尽量提高系统吞吐量,最后通过系统实现与实验测试对其测试及传输性能进行评估。本文首先提出基于动态滑动窗口平均的PDR测试算法,通过数据包收发事件驱动确定采样间隔,从而避免多配置对PDR测试的影响,同时利用当前PDR信息降低测试开销。然后以PDR测试算法为基础,提出在线PDR-RSS建模框架,通过PDR-RSS模型数据库结合实时更新,解决MIMO-OFDM系统多配置带来的过渡窗口问题。以上的PDR测试算法与在线建模框架,通过同时应用物理层指标RSS与链路层指标PDR,有效地解决了移动性及多配置对链路质量测试与建模的影响。最后通过速率控制算法设计及其系统实现,对以上算法进行实验评估,评估结果表明本文提出的PDR测试算法能够提升89\%的测试精度,同时结合在线建模框架能够在不同MIMO配置下实现40\%的吞吐量提升。



